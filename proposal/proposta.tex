%%%%%%%%%%%%%%%%%%%%%%%%%%%%%%%%%%%%%%%%%%%%%%%%%%%%%%%%%%%%%%%%%%%%%%%%%%%%%%%%%%%%%%
% Modelo de Proposta de TCCs de Especialização
% Preparado por Leandro Krug Wives a partir do 
% pacote ii-ufrgs.zip disponivel em http://www.inf.ufrgs.br/utug
%%%%%%%%%%%%%%%%%%%%%%%%%%%%%%%%%%%%%%%%%%%%%%%%%%%%%%%%%%%%%%%%%%%%%%%%%%%%%%%%%%%%%%

%%%%%%%%%%%%%%%%%%%%%%%%%%%%%%%%%%%%%%%%%%%%%%%%%%%%%%%%%%%%%%%%%%%%%%%%%%%%%%%%%%%%%%
% Definição do tipo / classe de documento e estilo usado
%%%%%%%%%%%%%%%%%%%%%%%%%%%%%%%%%%%%%%%%%%%%%%%%%%%%%%%%%%%%%%%%%%%%%%%%%%%%%%%%%%%%%%
%
\documentclass[prop-esp]{iiufrgs}

%%%%%%%%%%%%%%%%%%%%%%%%%%%%%%%%%%%%%%%%%%%%%%%%%%%%%%%%%%%%%%%%%%%%%%%%%%%%%%%%%%%%%%
% Importação de pacotes
%%%%%%%%%%%%%%%%%%%%%%%%%%%%%%%%%%%%%%%%%%%%%%%%%%%%%%%%%%%%%%%%%%%%%%%%%%%%%%%%%%%%%%
% (a A seguir podem ser importados os pacotes necessários para o documento, de acordo 
% com a necessidade)
%
\usepackage[brazilian]{babel}	    % para texto escrito em pt-br
\usepackage[utf8]{inputenc}         % pacote para acentuação
\usepackage{graphicx}         	    % pacote para importar figuras
\usepackage[T1]{fontenc}            % pacote para conj. de caracteres correto
\usepackage{times}                  % pacote para usar fonte Adobe Times
\usepackage{enumerate}              % para lista de itens com letras
\usepackage{breakcites}
\usepackage{titlesec}
\usepackage{enumitem}
\usepackage{titletoc}               
%\usepackage{fixltx2e}              % para subscript
%\usepackage{amsmath}               % para \epsilon e matemática
%\usepackage{amsfonts}
%\usepackage{listings}			    % para listagens de código-fonte
%\usepackage{color}				    % para imagens e outras coisas coloridas
%\usepackage{mathptmx}              % p/ usar fonte Adobe Times nas formulas matematicas
%\usepackage{url}                   % para formatar URLs
%\usepackage[table,xcdraw]{xcolor}  % para tabelas coloridas
%\usepackage{longtable}             % para tabelas compridas (mais de uma página)
%\usepackage{float}
%\usepackage{booktabs}
%\usepackage{tabularx}
%\usepackage[breaklinks]{hyperref}
%\usepackage{caption} 

\usepackage[alf,abnt-emphasize=bf]{abntex2cite}	% pacote para usar citações abnt


%%%%%%%%%%%%%%%%%%%%%%%%%%%%%%%%%%%%%%%%%%%%%%%%%%%%%%%%%%%%%%%%%%%%%%%%%%%%%%%%%%%%%%
% Macros, ajustes e definições
%%%%%%%%%%%%%%%%%%%%%%%%%%%%%%%%%%%%%%%%%%%%%%%%%%%%%%%%%%%%%%%%%%%%%%%%%%%%%%%%%%%%%%
%

% define estilo de parágrafo para citação longa direta:
\newenvironment{citacao}{
    %\singlespacing
    %\footnotesize
    \small
    \begin{list}{}{
        \setlength{\leftmargin}{4.0cm}
        \setstretch{1}
        \setlength{\topsep}{1.2cm}
        \setlength{\listparindent}{\parindent}
    }
    \item[]}{\end{list}
}

% adiciona a fonte em figuras e tabelas
\newcommand{\fonte}[1]{\\Fonte: {#1}}

% Ative o seguinte caso alguma nota de rodapé fique muito longa e quebre entre múltiplas
% páginas
%\interfootnotelinepenalty=10000

%%%%%%%%%%%%%%%%%%%%%%%%%%%%%%%%%%%%%%%%%%%%%%%%%%%%%%%%%%%%%%%%%%%%%%%%%%%%%%%%%%%%%%
% Informações gerais                                   
%%%%%%%%%%%%%%%%%%%%%%%%%%%%%%%%%%%%%%%%%%%%%%%%%%%%%%%%%%%%%%%%%%%%%%%%%%%%%%%%%%%%%%

% título
\title{Otimizando StarVZ para carga de grandes volumes de dados} 

% autor
\author{Miyazaki}{Alexandre K. S.}

% orientador
\advisor[Prof.~Dr.]{Schnorr}{Lucas M.} 

% coorientador (se houver)
%\coadvisor[Prof.~Dr.]{Sobrenome}{Nome}

% local da realização (defesa) do trabalho 
\location{Porto Alegre}{RS} 

% data da realização (defesa) do trabalho (mês e ano)
\date{05}{2019}

% nome do curso 
\course{Curso de Especialização em Big Data \& Data Science}

% Palavras chave
\keyword{Palavra-chave1}
\keyword{Palavra-chave2}
\keyword{Palavra-chave3}


%%%%%%%%%%%%%%%%%%%%%%%%%%%%%%%%%%%%%%%%%%%%%%%%%%%%%%%%%%%%%%%%%%%%%%%%%%%%%%%%%%%%%%
% Início do documento e elementos pré-textuais
%%%%%%%%%%%%%%%%%%%%%%%%%%%%%%%%%%%%%%%%%%%%%%%%%%%%%%%%%%%%%%%%%%%%%%%%%%%%%%%%%%%%%%

% Declara início do documento
\begin{document}

% inclui folha de rosto 
\maketitle      

\selectlanguage{brazilian}


%%%%%%%%%%%%%%%%%%%%%%%%%%%%%%%%%%%%%%%%%%%%%%%%%%%%%%%%%%%%%%%%%%%%%%%%%%%%%%%%%%%%%
% Introdução
%
\chapter{Introdução} \label{intro}

A quantidade de dados gerados em todo o mundo diariamente é surpreendente. De acordo com \cite{ref:data_minute2}, há uma estimativa
que em 2020 cada pessoa deve gerar em média 1,7MB de dados. Estes frequentemente precisam ser processados para ter algum valor e,
com essa frequente necessidade de poder de computação, plataformas de HPC (\emph{High-Performance Computing}) evoluíram para utilizar algumas
tecnologias, como processadores multicore e GPUs (\emph{Graphics processing units}, comumente referenciadas como \emph{accelerators}).

Com a evolução do Hardware, a abordagem de desenvolvimento de aplicações de HPC tradicional, chamada de Bulk-synchronous parallel (BSP) tornou-se 
obsoleta, pois espera que os recursos de computação sejam homogêneos (nós idênticos, conectados por links estáveis e de alta vazão) e portanto, é incapaz de utilizar a heterogeneidade do ambiente ao seu favor. Uma abordagem que está sendo utilizada é desenvolver a aplicação em alto nível, descrevendo as computações como um Directed Acyclic Graph (DAG) de tarefas. Essa abordagem é implementada por múltiplos modelos de programação: OpenMP 4 \cite{ref:openmp4}, StarPU \cite{ref:starpu}, OmpSs \cite{ref:ompss}, ParSEC \cite{ref:parsec}, etc. 

Nesses modelos de programação, a responsabilidade de escalonar e executar a aplicação de forma performática é atribuída para outra camada de software, denominada \emph{runtime}. Ferramentas que antes auxiliavam o desenvolvedor da aplicação a fazer otimizações, como Paraver \cite{ref:paraver} e Vampir \cite{ref:vampir} não são mais suficientes.

Com essa motivação, foi desenvolvido um \emph{framework}, denominado StarVZ, cujo objetivo era fornecer uma visualização de dados mais elaborada, provendo entendimento e identificação facilitada de problemas de performance que não seriam identificados tão facilmente com abordagens de visualização clássicas. Esse \emph{framework} foi desenvolvido combinando \emph{pj\_dump},  a linguagem R \cite{ref:rlanguage} e algumas bibliotecas expressivas dessa linguagem (\emph{ggplot2} \cite{ref:ggplot2}, \emph{lpSolve} \cite{ref:lpsolve} e \emph{tidyverse} \cite{ref:tidyverse}), Org-mode \cite{ref:org-mode} e \emph{plotly} para análise de traces gerados pelo modelo de programação StarPU.

Foram realizados alguns estudos de caso com o StarVZ \cite{}, onde obtiveram-se compreendimentos de problemas de performance sutis do StarPU. Durante
esses estudos de caso, uma das fases de processamento levou cerca de 32 minutos para processar 18 GB de dados. Essa fase deve ser melhorada pois pode inviabilizar a análise de volumes maiores.


%%%%%%%%%%%%%%%%%%%%%%%%%%%%%%%%%%%%%%%%%%%%%%%%%%%%%%%%%%%%%%%%%%%%%%%%%%%%%%%%%%%%%
% FUNDAMENTAÇÃO TEÓRICA
%
\chapter{Fundamentação Teórica}

Embasamento teórico de sustentação do trabalho e que determinará sua abordagem. Citar e explicar trabalhos, teorias e estudos que embasem as análises a serem realizadas. 


%%%%%%%%%%%%%%%%%%%%%%%%%%%%%%%%%%%%%%%%%%%%%%%%%%%%%%%%%%%%%%%%%%%%%%%%%%%%%%%%%%%%%
% OBJETIVOS
%
\chapter{Objetivos}

\noindent \textbf{Objetivo Principal} 

O principal objetivo deste trabalho é otimizar o tempo de processamento da .


%%%%%%%%%%%%%%%%%%%%%%%%%%%%%%%%%%%%%%%%%%%%%%%%%%%%%%%%%%%%%%%%%%%%%%%%%%%%%%%%%%%%%
% METODOLOGIA
%
\chapter{Metodologia}

Descrever como o trabalho vai ser realizado. Consiste em descrever e detalhar as atividades e os métodos, técnicas e ferramentas computacionais a serem utilizadas em cada uma delas. 

Informar tipo de pesquisa, procedimentos, atividades a serem desenvolvidas.

É importante descrever como dados serão coletados, analisados e interpretados, quando o trabalho envolver tal tipo de atividade.

%%%%%%%%%%%%%%%%%%%%%%%%%%%%%%%%%%%%%%%%%%%%%%%%%%%%%%%%%%%%%%%%%%%%%%%%%%%%%%%%%%%%%
% CRONOGRAMA
%
\chapter{Cronograma}

\noindent Definir claramente as etapas de desenvolvimento do trabalho e seus prazos.

\noindent 

\noindent Sugest\~{a}o: utilizar um quadro como o seguinte:

\noindent 

\begin{tabular}{|p{1.8in}|p{0.3in}|p{0.3in}|p{0.3in}|p{0.3in}|p{0.3in}|p{0.3in}|p{0.3in}|} \hline 
Etapa & \multicolumn{7}{|p{2.0in}|}{Meses} \\ \hline 
 & \multicolumn{2}{|p{0.6in}|}{2018} & \multicolumn{5}{|p{1.4in}|}{2019} \\ \hline 
 & Nov & Dez & Jan & Fev & Mar & Abr & Mai \\ \hline 
 &  &  &  &  &  &  &  \\ \hline 
 &  &  &  &  &  &  &  \\ \hline 
 &  &  &  &  &  &  &  \\ \hline 
 &  &  &  &  &  &  &  \\ \hline 
 &  &  &  &  &  &  &  \\ \hline 
 &  &  &  &  &  &  &  \\ \hline 
 &  &  &  &  &  &  &  \\ \hline 
\end{tabular}



%%%%%%%%%%%%%%%%%%%%%%%%%%%%%%%%%%%%%%%%%%%%%%%%%%%%%%%%%%%%%%%%%%%%%%%%%%%%%%%%%%%
% BIBLIOGRAFIA 
%%%%%%%%%%%%%%%%%%%%%%%%%%%%%%%%%%%%%%%%%%%%%%%%%%%%%%%%%%%%%%%%%%%%%%%%%%%%%%%%%%%
%
\chapter{BIBLIOGRAFIA}

\bibliographystyle{abntex2-alf}
\bibliography{biblio}

\end{document}
