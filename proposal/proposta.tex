%%%%%%%%%%%%%%%%%%%%%%%%%%%%%%%%%%%%%%%%%%%%%%%%%%%%%%%%%%%%%%%%%%%%%%%%%%%%%%%%%%%%%%
% Modelo de Proposta de TCCs de Especialização
% Preparado por Leandro Krug Wives a partir do 
% pacote ii-ufrgs.zip disponivel em http://www.inf.ufrgs.br/utug
%%%%%%%%%%%%%%%%%%%%%%%%%%%%%%%%%%%%%%%%%%%%%%%%%%%%%%%%%%%%%%%%%%%%%%%%%%%%%%%%%%%%%%

%%%%%%%%%%%%%%%%%%%%%%%%%%%%%%%%%%%%%%%%%%%%%%%%%%%%%%%%%%%%%%%%%%%%%%%%%%%%%%%%%%%%%%
% Definição do tipo / classe de documento e estilo usado
%%%%%%%%%%%%%%%%%%%%%%%%%%%%%%%%%%%%%%%%%%%%%%%%%%%%%%%%%%%%%%%%%%%%%%%%%%%%%%%%%%%%%%
%
\documentclass[prop-esp]{iiufrgs}

%%%%%%%%%%%%%%%%%%%%%%%%%%%%%%%%%%%%%%%%%%%%%%%%%%%%%%%%%%%%%%%%%%%%%%%%%%%%%%%%%%%%%%
% Importação de pacotes
%%%%%%%%%%%%%%%%%%%%%%%%%%%%%%%%%%%%%%%%%%%%%%%%%%%%%%%%%%%%%%%%%%%%%%%%%%%%%%%%%%%%%%
% (a A seguir podem ser importados os pacotes necessários para o documento, de acordo 
% com a necessidade)
%
\usepackage[brazilian]{babel}	    % para texto escrito em pt-br
\usepackage[utf8]{inputenc}         % pacote para acentuação
\usepackage{graphicx}         	    % pacote para importar figuras
\usepackage[T1]{fontenc}            % pacote para conj. de caracteres correto
\usepackage{times}                  % pacote para usar fonte Adobe Times
\usepackage{enumerate}              % para lista de itens com letras
\usepackage{breakcites}
\usepackage{titlesec}
\usepackage{enumitem}
\usepackage{titletoc}               
%\usepackage{fixltx2e}              % para subscript
%\usepackage{amsmath}               % para \epsilon e matemática
%\usepackage{amsfonts}
%\usepackage{listings}			    % para listagens de código-fonte
%\usepackage{color}				    % para imagens e outras coisas coloridas
%\usepackage{mathptmx}              % p/ usar fonte Adobe Times nas formulas matematicas
%\usepackage{url}                   % para formatar URLs
%\usepackage[table,xcdraw]{xcolor}  % para tabelas coloridas
%\usepackage{longtable}             % para tabelas compridas (mais de uma página)
%\usepackage{float}
%\usepackage{booktabs}
%\usepackage{tabularx}
%\usepackage[breaklinks]{hyperref}
%\usepackage{caption} 

\usepackage[alf,abnt-emphasize=bf]{abntex2cite}	% pacote para usar citações abnt


%%%%%%%%%%%%%%%%%%%%%%%%%%%%%%%%%%%%%%%%%%%%%%%%%%%%%%%%%%%%%%%%%%%%%%%%%%%%%%%%%%%%%%
% Macros, ajustes e definições
%%%%%%%%%%%%%%%%%%%%%%%%%%%%%%%%%%%%%%%%%%%%%%%%%%%%%%%%%%%%%%%%%%%%%%%%%%%%%%%%%%%%%%
%

% define estilo de parágrafo para citação longa direta:
\newenvironment{citacao}{
    %\singlespacing
    %\footnotesize
    \small
    \begin{list}{}{
        \setlength{\leftmargin}{4.0cm}
        \setstretch{1}
        \setlength{\topsep}{1.2cm}
        \setlength{\listparindent}{\parindent}
    }
    \item[]}{\end{list}
}

% adiciona a fonte em figuras e tabelas
\newcommand{\fonte}[1]{\\Fonte: {#1}}

% Ative o seguinte caso alguma nota de rodapé fique muito longa e quebre entre múltiplas
% páginas
%\interfootnotelinepenalty=10000

%%%%%%%%%%%%%%%%%%%%%%%%%%%%%%%%%%%%%%%%%%%%%%%%%%%%%%%%%%%%%%%%%%%%%%%%%%%%%%%%%%%%%%
% Informações gerais                                   
%%%%%%%%%%%%%%%%%%%%%%%%%%%%%%%%%%%%%%%%%%%%%%%%%%%%%%%%%%%%%%%%%%%%%%%%%%%%%%%%%%%%%%

% título
\title{Otimizando StarVZ para carga de grandes volumes de dados} 

% autor
\author{Miyazaki}{Alexandre K. S.}

% orientador
\advisor[Prof.~Dr.]{Schnorr}{Lucas M.} 

% coorientador (se houver)
%\coadvisor[Prof.~Dr.]{Sobrenome}{Nome}

% local da realização (defesa) do trabalho 
\location{Porto Alegre}{RS} 

% data da realização (defesa) do trabalho (mês e ano)
\date{05}{2019}

% nome do curso 
\course{Curso de Especialização em Big Data & Data Science} 

% Palavras chave
\keyword{Palavra-chave1}
\keyword{Palavra-chave2}
\keyword{Palavra-chave3}


%%%%%%%%%%%%%%%%%%%%%%%%%%%%%%%%%%%%%%%%%%%%%%%%%%%%%%%%%%%%%%%%%%%%%%%%%%%%%%%%%%%%%%
% Início do documento e elementos pré-textuais
%%%%%%%%%%%%%%%%%%%%%%%%%%%%%%%%%%%%%%%%%%%%%%%%%%%%%%%%%%%%%%%%%%%%%%%%%%%%%%%%%%%%%%

% Declara início do documento
\begin{document}

% inclui folha de rosto 
\maketitle      

\selectlanguage{brazilian}


%%%%%%%%%%%%%%%%%%%%%%%%%%%%%%%%%%%%%%%%%%%%%%%%%%%%%%%%%%%%%%%%%%%%%%%%%%%%%%%%%%%%%
% Introdução
%
\chapter{Introdução} \label{intro}




Deve contextualizar o trabalho do aluno, indicando o tema e a problemática (motivo, dúvida, questão ou problema a ser resolvido e sua importância dentro do contexto descrito). 

Deve abordar ou aludir a trabalhos anteriores e referências sobre o assunto, demonstrando o conhecimento do aluno na área e auxiliando o leitor a identificar o quanto o proponente conhece sobre o contexto e o problema a ser investigado e quais são as fundamentações teóricas (em termos mais gerais) que ele irá utilizar. 

Deve ainda informar as principais contribuições (esperadas) do trabalho, mostrando o quão relevante ele é. 


%%%%%%%%%%%%%%%%%%%%%%%%%%%%%%%%%%%%%%%%%%%%%%%%%%%%%%%%%%%%%%%%%%%%%%%%%%%%%%%%%%%%%
% FUNDAMENTAÇÃO TEÓRICA
%
\chapter{Fundamentação Teórica}

Embasamento teórico de sustentação do trabalho e que determinará sua abordagem. Citar e explicar trabalhos, teorias e estudos que embasem as análises a serem realizadas. 


%%%%%%%%%%%%%%%%%%%%%%%%%%%%%%%%%%%%%%%%%%%%%%%%%%%%%%%%%%%%%%%%%%%%%%%%%%%%%%%%%%%%%
% OBJETIVOS
%
\chapter{Objetivos}

\noindent \textbf{Objetivo Principal} 

Descrever o ponto principal do trabalho (usar infinitivo). \\


\noindent \textbf{Objetivos Secundários} 

Descrever ideias específicas, metas a serem atingidas para resolver o objetivo principal (usar infinitivo).


%%%%%%%%%%%%%%%%%%%%%%%%%%%%%%%%%%%%%%%%%%%%%%%%%%%%%%%%%%%%%%%%%%%%%%%%%%%%%%%%%%%%%
% METODOLOGIA
%
\chapter{Metodologia}

Descrever como o trabalho vai ser realizado. Consiste em descrever e detalhar as atividades e os métodos, técnicas e ferramentas computacionais a serem utilizadas em cada uma delas. 

Informar tipo de pesquisa, procedimentos, atividades a serem desenvolvidas.

É importante descrever como dados serão coletados, analisados e interpretados, quando o trabalho envolver tal tipo de atividade.

%%%%%%%%%%%%%%%%%%%%%%%%%%%%%%%%%%%%%%%%%%%%%%%%%%%%%%%%%%%%%%%%%%%%%%%%%%%%%%%%%%%%%
% CRONOGRAMA
%
\chapter{Cronograma}

\noindent Definir claramente as etapas de desenvolvimento do trabalho e seus prazos.

\noindent 

\noindent Sugest\~{a}o: utilizar um quadro como o seguinte:

\noindent 

\begin{tabular}{|p{1.8in}|p{0.3in}|p{0.3in}|p{0.3in}|p{0.3in}|p{0.3in}|p{0.3in}|p{0.3in}|} \hline 
Etapa & \multicolumn{7}{|p{2.0in}|}{Meses} \\ \hline 
 & \multicolumn{2}{|p{0.6in}|}{2018} & \multicolumn{5}{|p{1.4in}|}{2019} \\ \hline 
 & Nov & Dez & Jan & Fev & Mar & Abr & Mai \\ \hline 
 &  &  &  &  &  &  &  \\ \hline 
 &  &  &  &  &  &  &  \\ \hline 
 &  &  &  &  &  &  &  \\ \hline 
 &  &  &  &  &  &  &  \\ \hline 
 &  &  &  &  &  &  &  \\ \hline 
 &  &  &  &  &  &  &  \\ \hline 
 &  &  &  &  &  &  &  \\ \hline 
\end{tabular}



%%%%%%%%%%%%%%%%%%%%%%%%%%%%%%%%%%%%%%%%%%%%%%%%%%%%%%%%%%%%%%%%%%%%%%%%%%%%%%%%%%%
% BIBLIOGRAFIA 
%%%%%%%%%%%%%%%%%%%%%%%%%%%%%%%%%%%%%%%%%%%%%%%%%%%%%%%%%%%%%%%%%%%%%%%%%%%%%%%%%%%
%
\chapter{BIBLIOGRAFIA}

\bibliographystyle{abntex2-alf}

Listagem (em ordem alfabética) de todos os materiais que foram utilizados como base para elaborar a proposta, além dos documentos que serão utilizados no desenvolvimento da monografia.

Seguem alguns exemplos (retirados do modelo de monografias da biblioteca do Instituto de Informática da UFRGS):

\noindent {\bf \underbar{Monografia no todo}}\\

\noindent {\bf Livros e Anais de Congresso (Autor. Título. Edição. Local de Publicação: editora, ano de publicação).}\\

\noindent FURASTÉ, Pedro Augusto. {\bf Normas Técnicas para o Trabalho Científico}: explicitação das normas da ABNT. Porto Alegre: [s.n.], 2002. p. 49-56.

\noindent BRADLEY, N. {\bf The XML Companion}. 3${}^{rd}$ ed. Boston: Addison-Wesley, 2002.

\noindent FIELDS, D. K.; KDLB, M. A. {\bf Desenvolvendo na Web com JavaServer Pages}. Rio de Janeiro: Ciência Moderna, 2000.

\noindent OLIVEIRA, R. S. de; CARISSIMI, A. da S.; TOSCANI, S. S. {\bf Sistemas Operacionais}. 2.ed. Porto Alegre: Instituto de Informática da UFRGS: Sagra Luzzatto, 2001. 247 p. (Série Livros Didáticos, n.11).

\noindent SIMPÓSIO BRASILEIRO DE SISTEMAS MULTIMÍDIA E HIPERMÍDIA, SBMÍDIA, 7., 2001, Florianópolis. ... Florianópolis: UFSC: SBC, 2001.

\noindent NATIONAL CONFERENCE ON ARTIFICIAL INTELLIGENCE, AAII, 17., 2000. {\bf Proceedings}... Menlo Park, CA: AAAI Press: The MIT Press, 2000.

\noindent ~

\noindent {\bf \underbar{Parte de Monografia}}\\

\noindent {\bf Capítulo (Autor do capítulo. Título do capítulo. In: Autor/Editor/Organizador do livro. Título do livro. Edição. Local de publicação: editora, ano de publicação).}

\noindent LUBASZEWSKI, M.; COTA, E. F.; KRUG, M. R. Teste e Projeto Visando o Teste de Circuitos e Sistemas Integrados. In: REIS, R. A. da L. (Ed.) {\bf Concepção de Circuitos Integrados}. 2.ed. Porto Alegre: Instituto de Informática da UFRGS: Sagra Luzzatto, 2002. p. 167-189.

\noindent ROESLER, V.; BRUNO, G. G.; LIMA, J. V. de. ALM: Adaptative Layering Multicast. In: SIMPÓSIO BRASILEIRO DE SISTEMAS MULTIMÍDIA, SBMÍDIA, 7., 2001, Florianópolis. {\bf Anais...} Florianópolis: UFSC: SBC, 2001. p. 107-121.

\noindent PFEFFER, A.; KOLLER, D. Semantics and Inference for Recursive Probability Models. In: NATIONAL CONFERENCE ON ARTIFICIAL INTELLIGENCE, AAII, 17., 2000. {\bf Proceedings... }Menlo Park, CA: AAAI Press: The MIT Press, 2000.

\noindent ~

\noindent {\bf \underbar{Dissertações, teses, trabalhos individuais, etc.}}\\

\noindent MENEGHETTI, E. A. {\bf Uma Proposta de Uso da Arquitetura Trace como um Sistema de Detecção de Intrusão}. 2002. 105 f. Dissertação ( Mestrado em Ciência da Computação ) -- Instituto de Informática, UFRGS, Porto Alegre.

\noindent SABADIN, R. da S. {\bf QoS em Serviços de Suporte por Frame Relay}. 2000. 35 f. Trabalho Individual ( Mestrado em Ciência da Computação ) -- Instituto de Informática, UFRGS, Porto Alegre.

\noindent OTERO, I. M. {\bf Desenvolvimento de Sistema Cliente-Servidor em Camadas Utilizando Software Livre}. 2003. 55 f. Projeto de Diplomação ( Bacharelado em Ciência da Computação ) -- Instituto de Informática, UFRGS, Porto Alegre.

\noindent ~

\noindent {\bf \underbar{Artigo de periódico}}\\

\noindent GONÇALVES, L. M. G.; CESAR JUNIOR, R. M. Robótica, Sistemas Sensorial e Motos: principais tendências e direções. {\bf Revista de Informática Teórica e Aplicada}, Porto Alegre, v.9, n.2, p. 7-36, out. 2002.

\noindent JANOWIAK, R. M. Computers and Communications: a symbiotic relationship. {\bf Computer}, New York, v.36, n.1, p. 76-79, Jan. 2003.

\noindent ~

\noindent {\bf \underbar{Em meio eletrônico}}\\

\noindent LISBOA FILHO, J.; IOCHPE, C.; BORGES, K. Reutilização de Esquemas de Bancos de Dados em Aplicações de Gestão Urbana. {\bf IP -- Informática Pública}, Belo Horizonte, v.4, n.1, p.105-119, June 2002. Disponível em: $<$http://www.ip.pbh.gov.br/ip0401.html $>$. Acesso em: set. 2002.

\noindent ~

\noindent {\bf \underbar{RFC}}\\

\noindent CALLAGHAN, B.; PAWLOWSKI, B.; STAUBACH, P. {\bf NFS Version 3 Protocol Specification}: RFC 1831. [S.l.]: Internet Engineering Task Force, Network Working Group, 1995.

\noindent ~

\noindent {\bf \underbar{Norma}}\\

\noindent INSTITUTE OF ELECTRICAL AND ELECTRONIC ENGINEERING. {\bf IEEE 1003.1c-1995}: information technology -- portable operating system interface (POSIX), threads extension [C language]. New York, 1995.

\noindent ~

\noindent {\bf \underbar{Observações}}\\

Quando existirem mais de três autores, indica-se apenas o primeiro, acrescentando-se a expressão et al. Ex.: URANI, A. et al. Em casos em que a menção dos nomes for indispensável para certificar a autoria é facultado indicar todos os nomes.

Em caso de autoria desconhecida, a entrada é feita pelo título. Ex.: DIAGNÓSTICO do Setor Editorial Brasileiro. São Paulo: Câmara Brasileira do Livro, 1993.

Quando houver uma indicação de edição, esta deve ser transcrita, utilizando-se abreviaturas dos numerais ordinais e da palavra edição, ambas na forma adotada na língua do documento.

Ex.: SCHAM, D. {\bf Schawm's Outline of Theory and Problems}. 5${}^{th}$ ed. New York: Schawm Publishing, 1956.

PEDROSA, I. {\bf Da Cor a Cor Inexistente}. 6. ed. Rio de Janeiro: L. Cristiano, 1995.

Não sendo possível determinar o local (cidade) de publicação, utiliza-se à expressão sine loco, abreviada, entre colchetes [S.l.].

Quando a editora não puder ser indicada, deve-se indicar a expressão sine nomine, abreviada, entre colchetes [s.n].

Quando o local e a editora não puderem ser identificados, utilizam-se [S.l.:s.n].



\end{document}
