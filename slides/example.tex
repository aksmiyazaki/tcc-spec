%% example.tex
%% Copyright 2012 Bruno Menegola
%
% This work may be distributed and/or modified under the
% conditions of the LaTeX Project Public License, either version 1.3
% of this license or (at your option) any later version.
% The latest version of this license is in
%   http://www.latex-project.org/lppl.txt
% and version 1.3 or later is part of all distributions of LaTeX
% version 2005/12/01 or later.
%
% This work has the LPPL maintenance status ‘maintained’.
%
% The Current Maintainer of this work is Bruno Menegola.
%
% This work consists of all files listed in MANIFEST
%
%
% Description
% ===========
%
% This is an example latex document to build presentation slides based on
% the beamer class using the Inf theme.

\documentclass{beamer}

\usepackage[T1]{fontenc}
\usepackage[brazil]{babel}
\usepackage[utf8]{inputenc}
\usepackage{graphicx}
\usepackage{booktabs}
\usepackage{palatino}
\usepackage{hyphenat}
\usepackage{listings}
\usepackage{caption}
\renewcommand{\raggedright}{\leftskip=0pt \rightskip=0pt plus 0cm}

% Choose the Inf theme
\usetheme{Inf}

% Define the title with \title[short title]{long title}
% Short title is optional
\title[]
      {Otimizando StarVZ para carga de grandes volumes de dados}

% Optional subtitle
\subtitle{Especialização em Big Data \& Data Science}

\date{Setembro de 2019}

% Author information
\author{Alexandre Miyazaki  \\ Orientador: Lucas Schnorr}
\institute{Instituto de Informática --- UFRGS\\\texttt{inf.ufrgs.br/\~{}bmenegola}}

\begin{document}

% Command to create title page
\InfTitlePage

\begin{frame}
  \frametitle{Agenda}
  \tableofcontents
\end{frame}

\section{Introdução}
\begin{frame}
\frametitle{StarVZ}
  \begin{itemize}
   \item StarVZ é um arcabouço de análise de desempenho cujo objetivo é auxiliar 
na verificação de hipóteses sobre aplicações baseadas em tarefas.
   \item Este domínio carece de ferramentas de visualização voltadas para este 
tipo de aplicação, devido a dominância do modelo predecessor (BSP - 
\textit{Bulk-Synchronous Parallel}).
  \item Ele é separado em duas fases sendo uma de pré-processamento e outra de 
visualização.
  \end{itemize}
\end{frame}

\begin{frame}
\frametitle{Motivação}
  \begin{itemize}
   \item Em estudos anteriores, analisou-se logs com 18 GB.
   \item A primeira fase do StarVZ levou cerca de 32 minutos para processar.
   \item Tal desempenho pode inviabilizar o uso do StarVZ para cargas de 
grandes volumes de dados.
  \end{itemize}
\end{frame}

\begin{frame}
 \frametitle{Objetivo}
\begin{itemize}
 \item Viabilizar o uso do StarVZ para análise de grandes volumes de dados. 
Para isso, iniciamos pela otimização da parte mais onerosa, o fluxo 
de processamento R.
 \item Isto deverá ser atingido utilizando ferramentas de processamento de 
grandes volumes de dados, como o Hadoop e o Spark.
\end{itemize}
\end{frame}


\section{Ferramentas}

\begin{frame}
 \frametitle{Hadoop}
 \begin{columns}[T] % align columns
  \begin{column}{.40\textwidth}
  \begin{itemize}
  \item Arcabouço que permite o processamento de grandes volumes de dados.
  \item Organizado em camadas.
  \end{itemize}
  \end{column}%
  \hfill%
  \begin{column}{.56\textwidth}
  \begin{figure}[H]
  \centerline{\includegraphics[width=1\textwidth]{./img/hadoop-layers.pdf}}
  \label{fig:hadoop}
  \end{figure}
  \end{column}%
  \end{columns}
\end{frame}

\begin{frame}
 \frametitle{Spark}
 \begin{columns}[T] % align columns
  \begin{column}{.40\textwidth}
  \begin{itemize}
  \item \textit{Engine} unificada + conjunto de bibliotecas para processamento 
de dados distribuídos.
  \item Utilizado via biblioteca \texttt{sparklyr}.
  \end{itemize}
  \end{column}%
  \hfill%
  \begin{column}{.56\textwidth}
  \begin{figure}[ht]
  \centerline{\includegraphics[width=1\textwidth]{./img/spark-arch.pdf}}
  \label{fig:spark-arch}
  \end{figure}
  \end{column}%
  \end{columns}
\end{frame}

\section{Implementação}
\begin{frame}
 \frametitle{Abordagem}
 \begin{itemize}
  \item Otimizar processamento de tabelas.
  \item Utilizar equivalências entre \texttt{sparklyr} e \texttt{dplyr}, pois o 
fluxo de processamento do StarVZ é implementado em R.
  \item Processo foi facilitado pois a \texttt{sparklyr} é baseada na {dplyr}.
  \item Validação realizada com um conjunto de dados pequeno (835 MB), 
comparação de desempenho realizada com uma carga de trabalho maior (12 GB).
 \end{itemize}
\end{frame}

\begin{frame}
 \frametitle{Fluxo da Aplicação}
 \begin{figure}[H]
 \centerline{\includegraphics[width=1\textwidth]{./img/applicationflow.pdf}}
 \label{fig:spark-starvz-flow}
 \end{figure}
\end{frame}

\begin{frame}
 \frametitle{Equivalências}
 \begin{table}[H]
  \centering
  \small
  \begin{tabular}{c c} \toprule
  \textbf{Operação \texttt{dplyr}}  &  \textbf{Operação \texttt{sparklyr}}\\ 
  \midrule
  distinct	& unique  \\
  sort		& sdf\_sort \\
  gsub		& regexp\_replace\\
  rbind		& union\_all\\
  grepl		& rlike\\
  separate	& ft\_regex\_tokenizer + sdf\_separate\_column       \\
  \end{tabular}
  \label{tab:equivalence}
  \end{table}
\end{frame}

\begin{frame}
 \frametitle{Validação}
 \begin{itemize}
  \item Realizada por tabela.
  \item Diversos agrupamentos realizados em suas colunas.
  \item Comparação de resultados entre execução sequencial e distribuída.
 \end{itemize}
\end{frame}


\section{Avaliação e Resultados}
\begin{frame}
 \frametitle{Arquitetura dos experimentos}
  \begin{figure}[ht]
  \centerline{
  \includegraphics[width=0.9\textwidth]{./img/experiments_arch.pdf}}
  \label{fig:experiment_arch}
  \end{figure}
\end{frame}

\begin{frame}
 \frametitle{Resultados}
 \begin{figure}[ht]
  \centerline{
  \includegraphics[width=0.8\textwidth]{./img/total.pdf}}
  \label{fig:total_full}
  \end{figure}
\end{frame}


\begin{frame}
 \frametitle{Resultados}
  \begin{columns}[T] % align columns
  \begin{column}{.40\textwidth}
  \begin{table}[ht]
  \tiny
  \begin{tabular}{l c c c c} \toprule
  \textbf{Etapa}  & \textbf{1} & \textbf{15} & \textbf{15+15} & 
  \textbf{15+15+15}\\ 
  \midrule
  State		& 873.31 & 215.93 & 142.15 & 119.20\\
  Variable  	& 210.77 & 21.17  & 10.44  & 7.50 \\
  Link      	& 8.93   & 4.59   & 3.84   & 3.53 \\
  DAG        	& 69.14  & 10.10  & 7.58   & 6.76 \\
  Entities	& 3.07   & 2.42   & 2.31   & 2.30 \\
  Events		& 89.87  & 42.68  & 22.65  & 15.91\\
  GAPS		& 71.51  & 110.83 & 110.83 & 95.22\\
  Write		& 162.19 & 211.06 & 125.40 & 102.87\\
  \end{tabular}
  \label{tab:total_step}
  \end{table}
  \end{column}%
  \hfill%
  \begin{column}{.56\textwidth}
  \begin{figure}[H]
  \centerline{
  \includegraphics[width=0.8\textwidth]{./img/total_step.pdf}}
  \label{fig:total_step}
  \end{figure}
  \end{column}%
  \end{columns}
\end{frame}



\section{Conclusão}
\begin{frame}
 \frametitle{Objetivo}
 \begin{itemize}
 \item Viabilizar o uso do StarVZ para análise de grandes volumes de dados. 
Para isso, iniciamos pela otimização da parte mais onerosa, o fluxo 
de processamento R.
 \item Isto deverá ser atingido utilizando ferramentas de processamento de 
grandes volumes de dados, como o Hadoop e o Spark.
 \end{itemize}
\end{frame}

\begin{frame}
 \frametitle{Realizações}
 \begin{itemize}
  \item Apenas com as equivalências entre \texttt{dplyr} e \texttt{sparklyr}, 
foi possível atingir \emph{speedups} totais de:	
    \begin{itemize}
      \item 2,10x com 1 nó / 15 executores;
      \item 3,23x com 2 nós / 30 executores;
      \item 3,86x com 3 nós / 45 executores.
    \end{itemize}
  \item Portanto, conseguimos otimizar a etapa mais custosa do arcabouço StaVZ.
  \item Vale salientar que com o formato de execução distribuído, o StarVZ é 
capaz de processar mais dados do que o tamanho de memória de uma única máquina.
 \end{itemize}
\end{frame}

\begin{frame}
 \frametitle{Trabalhos Futuros}
 \begin{itemize}
  \item Testes com grandes volumes de dados.
  \item Testes com arquivos de registro de outras aplicações.
  \item Acabamento ao pacote StarVZ.
  \item Otimização de demais etapas da fase de pré-processamento do StarVZ.
 \end{itemize}
\end{frame}


\section*{}
\begin{frame}
    \frametitle{Obrigado!}
    \InfContacts
\end{frame}

\end{document}



