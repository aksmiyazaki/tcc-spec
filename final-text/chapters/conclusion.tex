\chapter{Conclusão e Trabalhos Futuros} \label{ch:conclusion}

Este trabalho teve como objetivo otimizar o fluxo de Ciência de Dados realizado 
em R dentro da primeira fase (pré-processamento) do arcabouço StarVZ. 
Analisando trabalhos anteriores, identificamos que este seria o ponto de 
otimização mais interessante, tendo em vista que ele é o que mais contribui 
para esta fase, sendo responsável por 40,62\% do tempo total.

Nos testes realizados com uma carga de trabalho de 12 GB, observamos um tempo 
de 1489,02 segundos ($\approx$ 24 minutos). Como esta é apenas uma parcela do 
tempo total da fase de pré-processamento, isso pode inviabilizar seu uso para 
maiores volumes de dados.

Nas adaptações realizadas, foi utilizado o ferramental para processamento de 
grandes volumes de dados, como o Hadoop e o Spark. Usamos o HDFS para 
armazenamento de entradas e saídas e o Spark para tratamento dos dados. Para 
adaptar a aplicação ao Spark, foi utilizada a biblioteca \mytexttt{sparklyr}, o 
que facilitou muito a adaptação pois ela é inspirada no 
\mytexttt{dplyr}, pacote utilizado no fluxo original.

Os experimentos mostraram redução no tempo de execução. Com um nó, apenas 
utilizando o Spark para realizar o processamento paralelo, ele foi reduzido 
pela metade metade (2,10x de \emph{speedup}). Ao realizar o processamento de 
forma distribuída com 2 e 3 nós, atingimos \emph{speedups} de 3,23x e 3,86x 
respectivamente, no melhor caso, levando apenas 385,44 segundos. Embora esta 
seja apenas uma etapa da fase de pré-processamento, tal redução contribui para 
a viabilização de sua utilização com volumes de dados maiores.

Além disso, este fluxo é limitado pela quantidade física de memória presente 
nas máquinas. Ao utilizar o HDFS e o Spark, permitimos que a aplicação processe 
mais dados do que esta limitação, avanço importante para o processamento de 
grandes volumes de dados.

Como trabalhos futuros, dentro do fluxo de Ciência de Dados, é importante 
realizar mais avaliações (no final do trabalho foi disponibilizado uma entrada 
de centenas de Gigabytes, mas não tivemos tempo de realizar testes com ela). 
Diversificar as avaliações também é importante pois os rastros de outros tipos 
de aplicação (diferentes de cholesky) podem ter um comportamento diferente. 
Além disso, as demais ferramentas utilizadas nesta fase também precisam ser 
trabalhadas (\mytexttt{starpu\_fxt\_tool}, \mytexttt{dot2csv}, 
\mytexttt{pjdump}) para otimizar-se o pré-processamento total.