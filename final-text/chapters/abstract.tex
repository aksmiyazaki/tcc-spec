% resumo na língua do documento
\begin{abstract}
Este documento é um exemplo de como formatar documentos para o
Instituto de Informática da UFRGS usando as classes \LaTeX\
disponibilizadas pelo UTUG\@. Ao mesmo tempo, pode servir de consulta
para comandos mais genéricos. \emph{O texto do resumo não deve
conter mais do que 500 palavras.}
\end{abstract}

% resumo na outra língua
% como parametros devem ser passados o titulo e as palavras-chave
% na outra língua, separadas por vírgulas
\begin{englishabstract}{Using \LaTeX\ to Prepare Documents at II/UFRGS}{Electronic document preparation, \LaTeX, ABNT, UFRGS}
This document is an example on how to prepare documents at II/UFRGS
using the \LaTeX\ classes provided by the UTUG\@. At the same time, it
may serve as a guide for general-purpose commands. \emph{The text in
the abstract should not contain more than 500~words.}
\end{englishabstract}