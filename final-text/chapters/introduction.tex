% introducao
\chapter{Introdução} \label{ch:intro}

A quantidade de dados gerados em todo o mundo diariamente é surpreendente. De 
acordo com \citet{ref:data_minute2}, há uma estimativa que em 2020 para cada pessoa 
no mundo sejam produzidos em média média 1,7 MB de dados por segundo. Estes frequentemente precisam 
ser processados para ter algum valor, o que implica em uma constante necessidade 
por sistemas computacionais mais poderosos.

Plataformas de HPC (\textit{High-Performance Computing}) costumam ser o estado 
da arte em poder computacional. Tais ambientes evoluíram para utilizar 
uma variabilidade de recursos de hardware, como processadores multicore e GPUs 
(\textit{Graphics processing units}, comumente referenciadas como 
\textit{accelerators}). Com a evolução do hardware, a abordagem de 
desenvolvimento de aplicações de HPC tradicional, chamada de 
\textit{Bulk-synchronous parallel} (BSP) tornou-se obsoleta. Esta espera que os 
recursos de computação sejam homogêneos (idênticos), conectados por 
links estáveis e de alta vazão e portanto, é incapaz de utilizar a 
heterogeneidade dos recursos disponíveis no ambiente ao seu favor. 

A abordagem que está sendo utilizada para o desenvolvimento de aplicações nesse 
cenário é orientada a tarefas. Nela, a aplicação é desenvolvida em alto nível, 
descrevendo as computações como um \textit{Directed Acyclic Graph} (DAG). Ela é 
implementada em diverssos modelos de programação como OpenMP 4 
\cite{ref:openmp4}, StarPU \cite{ref:starpu}, OmpSs \cite{ref:ompss}, 
ParSEC \cite{ref:parsec}, etc. Neles a responsabilidade de 
escalonar e executar a aplicação de forma eficiente é atribuída para outra 
camada de software, denominada \textit{runtime}. Ferramentas de análise de 
rastros, que antes auxiliavam o desenvolvedor da aplicação a fazer 
otimizações, como Paraver \cite{ref:paraver} e Vampir \cite{ref:vampir} são 
pouco eficazes ao analisar aplicações baseadas em tarefas.

Com essa motivação, foi desenvolvido um arcabouço denominado StarVZ 
\cite{ref:starvz}, cujo objetivo é fornecer uma visualização de rastros mais 
elaborada, provendo facilidade no entendimento e identificação de problemas de 
desempenho sutis, que dificilmente seriam identificados com abordagens 
clássicas. Esse arcabouço foi desenvolvido combinando \textit{pj\_dump},  
a linguagem R \cite{ref:rlanguage}, algumas bibliotecas expressivas dessa 
linguagem (\mytexttt{ggplot2} \cite{ref:ggplot2}, \mytexttt{lpSolve} 
\cite{ref:lpsolve}, \mytexttt{tidyverse} \cite{ref:tidyverse}), Org-mode 
\cite{ref:org-mode} e \mytexttt{plotly} para análise de rastros gerados pelo 
modelo de programação StarPU.

Em um estudo de caso onde o StarVZ foi utilizado, ele conseguiu contribuir para 
a identificação de problemas de desempenho no StarPU. A primeira fase do 
arcabouço (pré-processamento de dados) levou cerca de 32 minutos para processar 
18 GB. Tendo em vista que a tendência é que aplicações gerem cada vez mais 
dados, o objetivo deste trabalho é otimizar essa etapa do arcabouço na tentativa 
de viabilizar sua utilização com volumes maiores de dados, com tempos e 
recursos aceitáveis.

Mais especificamente, este trabalho será focado na etapa de manipulações 
realizadas sobre tabelas utilizando R e a biblioteca \mytexttt{dplyr}, 
que são operações comumente realizadas em fluxos de Ciência de Dados. Esta 
é a etapa mais custosa da fase de pré-processamento, no estudo de caso citado, 
ela foi responsável por 13 dos 32 minutos (40,62\% do tempo total). 
Utilizaremos ferramentas de Big Data para atingir nossos objetivos, como o 
Hadoop \cite{ref:hadoopbook} e o Spark \cite{ref:sparkbook}, além das 
facilidades oferecidas pela biblioteca \mytexttt{sparklyr}. 

Observamos bons resultados nos experimentos comparativos da aplicação 
modificada em relação a original. No melhor caso, tivemos uma redução de 
74,11\% no tempo de execução ao processar uma entrada de 12 GB.

Este documento consiste em um trabalho de conclusão de curso da Especialização 
em Big Data \& Data Science. Os próximos Capítulos são organizados da seguinte 
forma: o Capítulo \ref{ch:fundamentation} apresenta a fundamentação necessária 
para o entendimento do trabalho; o Capítulo \ref{ch:starvz} descreve com 
detalhes o arcabouço StarVZ; o Capítulo \ref{ch:contribution} disserta sobre as 
modificações propostas e realizadas; o Capítulo \ref{ch:evaluation} apresenta 
os resultados obtidos; e o Capítulo \ref{ch:conclusion} finaliza este trabalho e 
cita algumas possibilidades de trabalhos futuros.
